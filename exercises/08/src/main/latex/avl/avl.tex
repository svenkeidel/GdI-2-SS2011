\documentclass[a4paper]{article}
\usepackage[ngerman]{babel}
\usepackage[utf8]{inputenc}
\usepackage{lmodern}
\usepackage{graphicx}
\graphicspath{{../../graphviz/}}
\DeclareGraphicsExtensions{.png}
\usepackage{listings}
\usepackage[lined, boxed, commentsnumbered]{algorithm2e}

\lstset{numbers=left, 
		frame=single,
		basicstyle=\footnotesize,
		tabsize=4}

\begin{document}

\section*{H 8.4 Operationen auf AVL-Bäumen}

\subsection*{a)}

Der AVL-Baum ist durch einfügen des Knotens 75 entstanden. Das Balance
Kriterium wird durch eine LR-Rotation mit w=79 wieder hergestellt.
Siehe Abbildung Origianl auf der Seite weiter unten.

\subsection*{b)}

\begin{figure}[!h]
	\begin{center}
		\includegraphics[scale=0.4]{avl_00.png}
	\end{center}
	\caption{Original}
	\label{fig:original}
\end{figure}

\begin{figure}[!h]
	\begin{center}
		\includegraphics[scale=0.4]{avl_i1_00.png}
	\end{center}
	\caption{Einfügen: 20, keine Rotation}
	\label{fig:avl_i1_00}
\end{figure}

\begin{figure}[!h]
	\begin{center}
		\includegraphics[scale=0.4]{avl_i2_00.png}
	\end{center}
	\caption{Einfügen: 8, vor RL-Rotation mit w=6}
	\label{fig:avl_i2_00}
\end{figure}

\begin{figure}[!h]
	\begin{center}
		\includegraphics[scale=0.4]{avl_i2_01.png}
	\end{center}
	\caption{Einfügen: 8, nach RL-Rotation mit w=6}
	\label{fig:avl_i2_01}
\end{figure}

\begin{figure}[!h]
	\begin{center}
		\includegraphics[scale=0.4]{avl_ii1_00.png}
	\end{center}
	\caption{Löschen: 47}
	\label{fig:avl_ii1_00}
\end{figure}

\begin{figure}[!h]
	\begin{center}
		\includegraphics[scale=0.4]{avl_ii1_01.png}
	\end{center}
	\caption{Löschen: 47, keine Rotation}
	\label{fig:avl_ii1_01}
\end{figure}

\begin{figure}[!h]
	\begin{center}
		\includegraphics[scale=0.4]{avl_ii2_00.png}
	\end{center}
	\caption{Löschen: 30}
	\label{fig:avl_ii2_00}
\end{figure}

\begin{figure}[!h]
	\begin{center}
		\includegraphics[scale=0.4]{avl_ii2_01.png}
	\end{center}
	\caption{Löschen: 30, vor LL-Rotation mit w=29}
	\label{fig:avl_ii2_01}
\end{figure}

\begin{figure}[!h]
	\begin{center}
		\includegraphics[scale=0.4]{avl_ii2_02.png}
	\end{center}
	\caption{Löschen: 30, nach LL-Rotation mit w=29}
	\label{fig:avl_ii2_02}
\end{figure}

\begin{figure}[!h]
	\begin{center}
		\includegraphics[scale=0.4]{avl_ii3_00.png}
	\end{center}
	\caption{Löschen: 78}
	\label{fig:avl_ii3_00}
\end{figure}

\begin{figure}[!h]
	\begin{center}
		\includegraphics[scale=0.4]{avl_ii3_01.png}
	\end{center}
	\caption{Löschen: 78, keine Rotation}
	\label{fig:avl_ii3_01}
\end{figure}

\begin{figure}[!h]
	\begin{center}
		\includegraphics[scale=0.4]{avl_ii4_00.png}
	\end{center}
	\caption{Löschen: 73}
	\label{fig:avl_ii4_00}
\end{figure}

\begin{figure}[!h]
	\begin{center}
		\includegraphics[scale=0.4]{avl_ii4_01.png}
	\end{center}
	\caption{Löschen: 73, keine Rotation}
	\label{fig:avl_ii4_01}
\end{figure}

\begin{figure}[!h]
	\begin{center}
		\includegraphics[scale=0.4]{avl_ii5_00.png}
	\end{center}
	\caption{Löschen: 68}
	\label{fig:avl_ii5_00}
\end{figure}

\begin{figure}[!h]
	\begin{center}
		\includegraphics[scale=0.4]{avl_ii5_01.png}
	\end{center}
	\caption{Löschen: 68, vor LR Rotation mit w=49}
	\label{fig:avl_ii5_01}
\end{figure}

\begin{figure}[!h]
	\begin{center}
		\includegraphics[scale=0.4]{avl_ii5_02.png}
	\end{center}
	\caption{Löschen: 68, nach LR Rotation mit w=49}
	\label{fig:avl_ii5_02}
\end{figure}

\clearpage
\section*{H 8.5 Algorithmen durchführen}
\subsection*{a)}

\lstinputlisting[language=Java,
	caption={Einfüge Reihenfolge in AVL-Baum ohne Rotation},
	captionpos=b,
	label=avlInsert]
	{../../../src/main/java/avl/avl.java}

\clearpage
Ausgabe:
\begin{verbatim}
Contents of queue in each iteration
i = 0	Q = [3, 7, 13, 15, 21, 33, 40, 47]
i = 1	Q = [3, 7, 13, 15, 33, 40, 47]
i = 2	Q = [3, 7, 13, 33, 40, 47]
i = 3	Q = [3, 7, 13, 40, 47]
i = 4	Q = [3, 7, 40, 47]
i = 5	Q = [3, 7, 47]
i = 6	Q = [3, 47]
i = 7	Q = [3]

Order = [21, 15, 33, 13, 40, 7, 47, 3, ]
\end{verbatim}

\subsection*{b)}

\begin{figure}[!h]
	\begin{center}
		\includegraphics[scale=0.4]{avl_b_01.png}
	\end{center}
	\caption{AVL-Bäume vor der Vereinigung}
	\label{fig:befor_union}
\end{figure}

\begin{figure}[!h]
	\begin{center}
		\includegraphics[scale=0.4]{avl_b_02.png}
	\end{center}
	\caption{AVL-Bäume nach der Vereinigung}
	\label{fig:after_union}
\end{figure}

\section*{H 8.6 Rotationsoperation}

\subsection*{a)}

\begin{algorithm}[H]
	\SetKwInOut{Input}{Eingabe}
	\SetKwInOut{Output}{Ausgabe}

	\SetKwFunction{parent}{Elternknoten}
	\SetKwFunction{left}{linker Nachfolger}
	\SetKwFunction{right}{rechter Nachfolger}
	\SetKwFunction{setleft}{setze linken Nachfolger}
	\SetKwFunction{setright}{setze rechten Nachfolger}
	\SetKwFunction{setparent}{setze Elternknoten}
	\SetKwData{W}{w}
	\SetKwData{E}{e}
	\SetKwData{LE}{u}
	\SetKwData{R}{r}

	\Input{Knoten w bei dem eine LL-Rotation durchgeführt soll}
	\Output{nichts}

	\BlankLine
	\E $\leftarrow$ \parent{\W}\;
	\LE $\leftarrow$ \left{\W}\;
	\R $\leftarrow$ \right{\LE}\;

	\BlankLine
	\tcp{setze Nachfolger von \E}

	\eIf{\left{\E} = \W}{
		\setleft{\E,\LE}\;
	}{
		\setright{\E,\LE}\;
	}

	\BlankLine
	\setleft{\W,\R}\;

	\BlankLine
	\setright{\LE, \W}\;
	
	\BlankLine
	\setparent{\LE, \E}\;
	\setparent{\W, \LE}\;
	\setparent{\R, \W}\;

	\caption{AVL-Rechtsrotation LL in Pseudocode}
\end{algorithm}

\clearpage
\subsection*{b)}

\begin{lstlisting}[
	caption={Einfüge- und Löschoperation für i)},
	captionpos=b,
	label=insertAndRemoveI]
remove(5);
remove(6);
insert(5);
insert(6);
remove(4);
insert(4);
\end{lstlisting}

\begin{lstlisting}[
	caption={Einfüge- und Löschoperation für ii)},
	captionpos=b,
	label=insertAndRemoveII]
remove(7);
remove(8);
remove(9);
insert(8);
insert(7);
insert(9);
\end{lstlisting}

\subsection*{c)}

Man muss teilweise ganze Teilbäume löschen um sie zu verschieben. Dies
ruft wesentlich mehr Elementar Operationen hervor als die Funktion aus
b).

\end{document}
