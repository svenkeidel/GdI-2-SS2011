\documentclass[a4paper]{article}
\usepackage[ngerman]{babel}
\usepackage[utf8]{inputenc}
\usepackage{verbatim}
\usepackage{array}
\usepackage{amsmath}
\usepackage{graphicx}
\graphicspath{{../../graphviz/}}
\DeclareGraphicsExtensions{.png}

\usepackage{listingsutf8}
\lstset{numbers=left, 
		basicstyle=\footnotesize,
		linewidth=\textwidth,
		tabsize=2,
		inputencoding=utf8/latin1}

\begin{document}

\section*{H 12.4 Geometrische Datenstrukturen}

\subsection*{a)}

\begin{figure}[!h]
	\begin{center}
		\includegraphics[scale=0.15]{quadtree01.png}
	\end{center}
	\caption{Quadtree der Mensa}
	\label{fig:qtree01}
\end{figure}

\subsection*{b)}

\begin{figure}[!h]
	\begin{center}
		\includegraphics[scale=0.2]{quadtree02.png}
	\end{center}
	\caption{Ergebnis des Schnitts der Quadtrees}
	\label{fig:qtree02}
\end{figure}

\clearpage
\subsection*{c)}

\begin{figure}[!h]
	\begin{center}
		\includegraphics[scale=0.6]{adaptiver_2d_tree.png}
	\end{center}
	\caption{Adaptiver 2d Tree der Eingänge der Mensa}
	\label{fig:adapt2dtree01}
\end{figure}

\subsection*{d)}

\subsubsection*{Vorangehensweise:}
\begin{itemize}
	\item Koordinaten in Binärdarstellung verwandeln.
	\item Bit Interleaving durchführen.
	\item Die Bits des Ergebnisses der vorherigen Operation sind die
		Bits entlang des Suchpfades des 2d Tries
\end{itemize}

\begin{tabular}[h!]{|c|c|c|}
	\hline
	Eingang & Binärdarstellung & nach Bit Interleaving \\
	\hline
	(1, 1) & $(1,1)_2 = (001,001)$ & 000011 \\
	(2, 3) & $(2,3)_2 = (010,011)$ & 001101 \\
	(4, 5) & $(4,5)_2 = (100,101)$ & 110001 \\
	(5, 3) & $(5,3)_2 = (101,011)$ & 100111 \\
	(6, 2) & $(6,2)_2 = (110,010)$ & 101100 \\
	(6, 4) & $(6,4)_2 = (110,100)$ & 111000 \\
	\hline
\end{tabular}

\clearpage
\section*{H 12.5 Hilbertkurve}

\subsection*{a)}

\lstinputlisting[language=Ruby,
                 caption={Hilbert.rb},
				 captionpos=b,
                 firstnumber=1,
				 linerange=1-32]
				 {../../../src/main/ruby/hilbert.rb}

\subsection*{b)}
TODO

\subsection*{c)}

\lstinputlisting[language=Ruby,
                 caption={Hilbert.rb},
				 captionpos=b,
                 firstnumber=34,
				 linerange=34-109]
				 {../../../src/main/ruby/hilbert.rb}

\subsection*{Ausgabe für Hilbert.to\_s(5)}

siehe Anhang

\end{document}
