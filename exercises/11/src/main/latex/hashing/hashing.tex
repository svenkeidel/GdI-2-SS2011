\documentclass[a4paper]{article}
\usepackage[ngerman]{babel}
\usepackage[utf8]{inputenc}
\usepackage{verbatim}
\usepackage{array}
\usepackage{amsmath}

\usepackage{listings}
\lstset{numbers=left, 
		basicstyle=\footnotesize,
		linewidth=\textwidth,
		tabsize=2}

\begin{document}

\section*{H 11.5 Hashing}

\lstinputlisting[language=Ruby,
                 firstnumber=0,
				 linerange={0-29}]
{../../../src/main/ruby/hash.rb}

\subsection*{a)}

\begin{verbatim}
 * insertion of key 1 at hash 1
 * insertion of key 3 at hash 3
 * insertion of key 5 at hash 5
 * insertion of key 9 at hash 9
 ! collision while insertion of key 11 at hash 1
 * insertion of key 11 at hash 2
 ! collision while insertion of key 13 at hash 3
 * insertion of key 13 at hash 4
 ! collision while insertion of key 15 at hash 5
 * insertion of key 15 at hash 6
 ! collision while insertion of key 19 at hash 9
 * insertion of key 19 at hash 0
 ! collision while insertion of key 10 at hash 0
 ! collision while insertion of key 10 at hash 1
 ! collision while insertion of key 10 at hash 2
 ! collision while insertion of key 10 at hash 3
 ! collision while insertion of key 10 at hash 4
 ! collision while insertion of key 10 at hash 5
 ! collision while insertion of key 10 at hash 6
 * insertion of key 10 at hash 7
Map       : [19, 1, 11, 3, 13, 5, 15, 10, nil, 9]
Fill      : 9
Collisions: 11
\end{verbatim}

\subsection*{b)}

\begin{verbatim}
 * insertion of key 1 at hash 1
 * insertion of key 3 at hash 3
 * insertion of key 5 at hash 5
 * insertion of key 9 at hash 9
 * insertion of key 11 at hash 0
 * insertion of key 13 at hash 2
 * insertion of key 15 at hash 4
 * insertion of key 19 at hash 8
 * insertion of key 10 at hash 10
Map       : [11, 1, 13, 3, 15, 5, nil, nil, 19, 9, 10]
Fill      : 9
Collisions: 0
\end{verbatim}

\subsection*{c)}

\subsubsection*{Einfüge-Reihenfolge für m=10}

\begin{verbatim}
 * insertion of key 0 at hash 0
 * insertion of key 2 at hash 2
 * insertion of key 4 at hash 4
 * insertion of key 6 at hash 6
 * insertion of key 8 at hash 8
Map       : [0, nil, 2, nil, 4, nil, 6, nil, 8, nil]
Fill      : 5
Collisions: 0
\end{verbatim}

\subsubsection*{Einfüge-Reihenfolge für m=11}

\begin{verbatim}
 * insertion of key 0 at hash 0
 * insertion of key 2 at hash 2
 * insertion of key 4 at hash 4
 * insertion of key 6 at hash 6
 * insertion of key 8 at hash 8
 * insertion of key 10 at hash 10
 * insertion of key 12 at hash 1
 * insertion of key 14 at hash 3
 * insertion of key 16 at hash 5
 * insertion of key 18 at hash 7
Map       : [0, 12, 2, 14, 4, 16, 6, 18, 8, nil, 10]
Fill      : 10
Collisions: 0
\end{verbatim}

\subsubsection*{Einfügen einer für m=11 erlaubten Reiheinfolge in m=10}

\begin{verbatim}
 * insertion of key 0 at hash 0
 * insertion of key 2 at hash 2
 * insertion of key 4 at hash 4
 * insertion of key 6 at hash 6
 * insertion of key 8 at hash 8
 ! collision while insertion of key 10 at hash 0
 * insertion of key 10 at hash 1
 ! collision while insertion of key 12 at hash 2
 * insertion of key 12 at hash 3
 ! collision while insertion of key 14 at hash 4
 * insertion of key 14 at hash 5
 ! collision while insertion of key 16 at hash 6
 * insertion of key 16 at hash 7
 ! collision while insertion of key 18 at hash 8
 * insertion of key 18 at hash 9
Map       : [0, 10, 2, 12, 4, 14, 6, 16, 8, 18]
Fill      : 10
Collisions: 5
\end{verbatim}

\subsection*{d)}

Die maximale Länge für m=10 ist 5 und für m=11 10.

\section*{H 11.6 Hashing}

\subsection*{a)}

Man braucht ein Array mit $10^7$ Einträgen, da es maximal $10^7$
verschiedene Matrikelnummern geben kann und alle gleichzeitig vergeben
werden könnten.

$1 - 23100 / 10^7 = 0,99769 = 99,769\%$

\subsection*{b)}

\begin{tabular}[h!]{|l| m{1.5cm} | m{2cm} |l|}
	\hline
	Schlüssel & Faltung & Shift Faltung & Midsqaremethode \\
	\hline
	101010    & 
	\begin{verbatim}
	  01
	  10
	 +01
	----
	  12
	%100
	----
	  12
	\end{verbatim}
	&
	\begin{verbatim}
	  10
	  10
	 +10
	----
	  30
	%100
	----
	  30
	\end{verbatim}
	&
	$101010^2 = 1020\textbf{30}20100$ \\

	\hline

	7081 &
	\begin{verbatim}
	  70
	 +18
	----
	  88
	%100
	----
	  88
	\end{verbatim}
	&
	\begin{verbatim}
	  70
	 +81
	----
	 151 
	%100
	----
	  51
	\end{verbatim}
	&
	$7081^2 = 501\textbf{40}561$ \\
	
	\hline

	999 &
	\begin{verbatim}
	  09
	 +99
	 ---
	 108
	%100
	 ---
	  08
	\end{verbatim}
	&
	\begin{verbatim}
	  09
	 +99
	 ---
	 108
	%100
	 ---
	  08
	\end{verbatim}
	&
	$999^2 = 99\textbf{80}01$ \\
	\hline
\end{tabular}

\subsection*{c)}

$f(x) = \frac{(x^2 \mod{10^5}) - (x^2 \mod{10^3})} {10^3}$

\subsection*{d)}

$f(x) = \frac{ x             - (x \mod{10^4})} {10^4} +
        \frac{(x \mod{10^4}) - (x \mod{10^2})} {10^2} + (x \mod{10^2})$

\clearpage
\section*{H 11.7 Externes Hashing}

\subsection*{a)}

Bucketfaktor b = 2

\subsection*{b)}

\begin{itemize}
	\item Durch 1 Einfügung in Position 0
	\item Durch 1 Einfügung in Position 1
	\item Durch 5 Einfügungen und 2 Löschungen in Position 2
	\item Durch 1 Einfügung in Position 3
	\item Durch 4 Einfügungen und 1 Löschung in Position 4
	\item Durch 3 Einfügungen und 1 Löschung in Position 5
	\item Durch 4 Einfügungen und 1 Löschung in Position 6
\end{itemize}

\subsection*{c)}

\lstinputlisting[language=Ruby]{../../../src/main/ruby/external_hash.rb}

\subsubsection*{Ausgabe:}

\begin{verbatim}
Mallorca   => 0
Sicilia    => 1
Jeju       => 2
Madagaskar => 2
Malediven  => 2
Jamaika    => 3
Srilanka   => 4
Bahamas    => 4
Hainau     => 4
Tahiti     => 5
Fidschi    => 5
Ibiza      => 6
Kuba       => 6
Island     => 6
Taiwan     => 4
\end{verbatim}

\subsection*{d)}
\begin{tabular}[!h]{|l|l|l|}
	\hline
	Index & Bucket 1 & Bucket 2 \\
	\hline
	\hline
	0 & Mallorca   &            \\
	\hline
	1 & Sicilia    &            \\
	\hline
	2 & Jeju       & Malediven  \\
	\hline
	3 & Jamaika    &            \\
	\hline
	4 & Srilanka   & Bahamas    \\
	  &            & Hainau     \\
	\hline
	5 & Fidschi    &            \\
	\hline
	6 & Ibiza      & Kuba       \\
	\hline
\end{tabular}

\end{document}
