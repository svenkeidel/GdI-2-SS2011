\documentclass[a4paper]{article}
\usepackage[ngerman]{babel}
\usepackage[utf8]{inputenc}
\usepackage{graphicx}
\graphicspath{{../../graphviz/}}
\DeclareGraphicsExtensions{.png}
\usepackage{listings}
\lstset{numbers=left, 
		frame=single,
		basicstyle=\footnotesize,
		tabsize=4}

\begin{document}

\section*{H 9.5 B-Bäume}

\subsection*{a)}

\begin{figure}[!h]
	\begin{center}
		\includegraphics[scale=0.4]{btree01.png}
	\end{center}
	\caption{Einfügen: 21}
	\label{fig:btree01}
\end{figure}

\begin{figure}[!h]
	\begin{center}
		\includegraphics[scale=0.4]{btree02.png}
	\end{center}
	\caption{Einfügen: 57}
	\label{fig:btree02}
\end{figure}

\begin{figure}[!h]
	\begin{center}
		\includegraphics[scale=0.4]{btree03.png}
	\end{center}
	\caption{Einfügen: 66, ausgleichen\ldots}
	\label{fig:btree03}
\end{figure}

\begin{figure}[!h]
	\begin{center}
		\includegraphics[scale=0.4]{btree04.png}
	\end{center}
	\caption{Einfügen: 66, ausgeglichen}
	\label{fig:btree04}
\end{figure}

\begin{figure}[!h]
	\begin{center}
		\includegraphics[scale=0.4]{btree05.png}
	\end{center}
	\caption{Einfügen: 67}
	\label{fig:btree05}
\end{figure}

\begin{figure}[!h]
	\begin{center}
		\includegraphics[scale=0.4]{btree06.png}
	\end{center}
	\caption{Einfügen: 3}
	\label{fig:btree06}
\end{figure}

\begin{figure}[!h]
	\begin{center}
		\includegraphics[scale=0.4]{btree07.png}
	\end{center}
	\caption{Einfügen: 13, ausgleichen\ldots}
	\label{fig:btree07}
\end{figure}

\begin{figure}[!h]
	\begin{center}
		\includegraphics[scale=0.4]{btree08.png}
	\end{center}
	\caption{Einfügen: 3, ausgeglichen}
	\label{fig:btree08}
\end{figure}

\begin{figure}[!h]
	\begin{center}
		\includegraphics[scale=0.4]{btree09.png}
	\end{center}
	\caption{Einfügen: 84, ausgleichen\ldots}
	\label{fig:btree09}
\end{figure}

\begin{figure}[!h]
	\begin{center}
		\includegraphics[scale=0.4]{btree10.png}
	\end{center}
	\caption{Einfügen: 84, ausgleichen\ldots, Eltern für 66, 84 noch unbestimmt}
	\label{fig:btree10}
\end{figure}

\begin{figure}[!h]
	\begin{center}
		\includegraphics[scale=0.4]{btree11.png}
	\end{center}
	\caption{Einfügen: 84, ausgeglichen}
	\label{fig:btree11}
\end{figure}

\clearpage
\subsection*{b)}

\begin{figure}[!h]
	\begin{center}
		\includegraphics[scale=0.4]{btree12.png}
	\end{center}
	\caption{Einfügen: 5}
	\label{fig:btree12}
\end{figure}

\begin{figure}[!h]
	\begin{center}
		\includegraphics[scale=0.4]{btree13.png}
	\end{center}
	\caption{Einfügen: 12, ausgleichen\ldots}
	\label{fig:btree13}
\end{figure}

\begin{figure}[!h]
	\begin{center}
		\includegraphics[scale=0.4]{btree14.png}
	\end{center}
	\caption{Einfügen: 12, ausgeglichen}
	\label{fig:btree14}
\end{figure}

\begin{figure}[!h]
	\begin{center}
		\includegraphics[scale=0.4]{btree15.png}
	\end{center}
	\caption{Löschen: 3, verschiebe 12 nach 5}
	\label{fig:btree15}
\end{figure}

\begin{figure}[!h]
	\begin{center}
		\includegraphics[scale=0.4]{btree16.png}
	\end{center}
	\caption{Löschen: 3, ausgeglichen}
	\label{fig:btree16}
\end{figure}

\begin{figure}[!h]
	\begin{center}
		\includegraphics[scale=0.4]{btree17.png}
	\end{center}
	\caption{Löschen: 21, verschiebe 12 zur 13 und 13 zur 21}
	\label{fig:btree17}
\end{figure}

\begin{figure}[!h]
	\begin{center}
		\includegraphics[scale=0.4]{btree18.png}
	\end{center}
	\caption{Löschen: 21, ausgeglichen}
	\label{fig:btree18}
\end{figure}

\begin{figure}[!h]
	\begin{center}
		\includegraphics[scale=0.4]{btree19.png}
	\end{center}
	\caption{Löschen: 67, verschiebe 66 zur 67 und 84 rechts daneben}
	\label{fig:btree19}
\end{figure}

\begin{figure}[!h]
	\begin{center}
		\includegraphics[scale=0.4]{btree20.png}
	\end{center}
	\caption{Löschen: 67, ausgeglichen}
	\label{fig:btree20}
\end{figure}

\clearpage
\subsection*{c)}

\begin{lstlisting}
public class BTree {

	// ...

	public void insert(BTree tree) {
		int order = tree.getOrder();

		if(tree.getCurrentNode().isLeaf()) {
			for(int i=0; i<order; i++)
				if(tree.getCurrentNode().hasKeyAt(i))
					this.addKey(tree.getCurrentNode().getKeyAt(i));
		}

		for(int i=0; i<order; i++) {
			if(tree.hasChildAt(i)) {
				tree.moveToChildAt(i);
				insert(tree);
				tree.moveToParent();
			}
		}
	}

	// ...

}

\end{lstlisting}

\end{document}
